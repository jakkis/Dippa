%\section{Johdanto}
\section{Introduction}
\label{sec:introduction}

%% Ensimmäinen sivu tyhjäksi
%% 
%% Leave first page empty
\thispagestyle{empty}

\begin{comment}
\begin{itemize}
\item[--]Tutkimuksen taustaa ja tutkimusaiheen yleisluonteinen esittely
\item[--]Tutkimuksen tavoitteet
\item[--]Pääkysymys ja osaongelmat
\item[--]Tutkimuksen rajaus ja keskeiset käsitteet.
\item[--]Työn rakenne
\end{itemize}
\end{comment}

The use of mobile devices have been growing rapidly for the last two decades, and the same trend will continue in the near future, as the global amount of mobile subscriptions is constantly increasing and applications are more and more data intensive \cite{nokia,cisco,ericsson,gsma}. Mobile phones are not anymore used for only traditional calls, but for communicating through a numerous different messaging services, social media, and other applications. Prior to the third generation mobile networks (3G), this have been impossible. 3G enlarged the channel bandwidth by factor of 25 \cite{molisch}, which enabled significantly higher data rates.

After 3G, the standards of mobile communications have kept on evolving, as new applications and demands have been invented. In 2008, International Telecommunications Union (ITU) announced the new requirements for mobile networks \cite{itur}. Long Term Evolution (LTE), or the fourth generation mobile networks (4G), would again multiply channel bandwidth and data rates to answer the need for faster networks. However, LTE does not fulfill the specifications, but its successor LTE-Advanced will. 4G networks use over 40 frequency bands around the world \cite{radio_electronics,molisch}, and provide 1\,Gb/s peak data transmission by utilizing techniques such as multiple input multiple output (MIMO) and carrier aggregation \cite{parkvall_lte}.

While the network standards have developed a lot, same time the mobile phones have changed their shapes several times \cite{anguera2}. Nowadays smartphones with large touch screens are the common device. Large displays reserve a major amount of space for themselves, which sets different constraints for the placement of other subsystems. Modern mobile phones operate already on a wide range of frequencies, and it is estimated that the amount of communications systems might exceed 20 in the future \cite{20ant}. This would mean 20 different antennas in a single device, which is quite a challenge for designers, as the network standards also demand a certain performance. To meet the requirements antennas should operate on multiple bands or support wideband communications \cite{lehtovuori_wideband_match}.

Fortunately, antenna techniques are developing together with the networks. From long monopoles and dipoles sticking outside the handset, the antennas are currently inside the phone, or integrated to the housing of the device \cite{saunders,molisch}. Different forms of planar antennas have been a popular choice, since they are able to operate on multiple bands \cite{anguera}. Although antennas are designed to operate on multiple bands, their size and shape might become uncontrollable, which restricts performance. A potential option is a capacitive coupling element (CCE), which can be used at a wide band, requires a smaller area, and typically has a simple structure \cite{valkonen_cce2}.

Modern networks and handheld devices ensure that mobility of people is not an obstacle for communicating with others anymore. Smart phones have become the main communication tools due to their rather compact size, light weight, and mechanical robustness. For improved strength and also better aesthetics, phone manufacturers have started to use metal covers and side frames \cite{rowell}. Metal covers and rims as conductive materials increase the size of the device's ground plane, which restricts the free space around the antennas. Close proximity of highly conductive materials also leads to narrower bandwidth and decreased efficiency \cite{rowell}. Antennas radiate poorly when enclosed by metal covers, as the structure resembles a small Faraday cage. To resolve this problem, phone manufacturers have started to integrate antennas straight into the metallic structures. For example, Apple's iPhone~4 among the first released metal-rimmed phones, has its antennas integrated into the rim. 

Only days after the release of iPhone~4, Apple received a lot of complaints about connectivity issues, and their first solution was to hold the phone differently \cite{apple_press,apple_bbc}. This solution clearly is not sustainable in the long run, and antenna designers are required to come up with new ideas for the future handsets. After iPhone~4, many other companies have released their metal-rimmed, or even metal-covered phones. Typical antenna has still been integrated to the structure, and it is usually separated from the other structures with a slot. In case of metal-covered phones, some slots are typically included to the back cover to improve the antenna performance. 

This thesis focuses on metal-covered phones. Moreover, the main purpose is to design antennas for mobile phone that has a slotless and continuous metal plate as a back cover. The thesis is part of a research project with AAC Technologies \cite{aac}, the specifications for the antennas are defined by the project. The detailed requirements for the antennas are specified in Section \ref{sec:objectives}. Briefly, the cellular antennas should be efficient, support MIMO and cover a wide range of frequencies. Additionally, the phone should have antennas for Wi-Fi (trademark for Wireless Local Area Networks (WLAN)), and Global Positioning System (GPS). In addition, this thesis studies the effect of the metal cover on antenna performance.

The structure of this thesis is as follows. In Section \ref{sec:mobile_antennas}, the required background knowledge is explained by introducing the characteristics of an antenna, methods to evaluate its performance, and presenting examples of different antenna structures. Section \ref{sec:metal_cover} reviews the latest studies on antenna structures for metal-covered handsets. The theoretical part is followed by \Cref{sec:objectives,sec:simulations,sec:analysis}, which present the main contribution by the author by explaining the research methods, presenting, and finally analyzing the results from the antenna simulations, respectively. At last, the thesis is concluded in Section \ref{sec:conclusions}.

%% Opinnäytteessä jokainen osa alkaa uudelta sivulta, joten \clearpage
%%
%% In a thesis, every section starts a new page, hence \clearpage
\clearpage
