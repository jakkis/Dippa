%\section{Yhteenveto}
\section{Summary and conclusions} 
\label{sec:conclusions}

\begin{comment}
\begin{itemize}
\item[--]yhteenveto mitä tehty
\item[--]päätulokset
\item[--]johtopäätökset
\item[--]oma arvio?
\end{itemize}
\end{comment}

In this thesis, antennas for metal-covered handsets have been studied. The work consists of two parts: literature review on previously studied antennas and a design project. The main objectives of this work are to design antennas for mobile terminal with unbroken metallic back cover, and to understand the challenges in the design process caused by the cover.

As the wireless networks have developed, and the volume of data traffic has simultaneously increased, more and more is required from the mobile terminal. Desires for better robustness and aesthetics by using metal covers significantly increase the complexity of the system in an antenna design point of view. The network requires antennas to communicate efficiently on a wide frequency band, but the metal structures disturb this. Traditional mobile phone antennas, e.g.\ PIFAs, placed inside the phone are highly affected by the surrounding conductive materials.

Majority of the previous studies on this topic have only a metallic side frame, and in the few studies that have also back cover, slots have been cut into it. Antenna structures proposed in the previous studies have had decent performance, but the mechanics of them are rather complex. This thesis differs from those studies as the back cover of the handset is a single continuous metal plate. Also, the designed antennas have quite simple structure, and they are integrable to the metal frame on the sides of the device.

Considering the challenging environment due to the metal cover, the performance targets defined by AAC Technologies are quite strict and challenging to achieve. The two cellular antennas are supposed to operate at $704-960\,\mega\hertz$ and $1.71-2.69\,\giga\hertz$, have at least 30\,\% efficiency, and be fully MIMO capable. Additionally, the two GPS/Wi-Fi antennas should have at least $40\,\%$ efficiency. As a remark, only a minority of previous studies support the $700-800\,\mega\hertz$ cellular band, or MIMO.

Designing the antennas is done in an electromagnetic simulator. The model of the phone is based on a mechanically accurate 3D-model of a real phone. This model is significantly more detailed than the models published in the previous studies. Nevertheless, a suitable antenna structure is constructed. The designed cellular antennas consist of three closely located elements, in order to have strong mutual coupling to increase bandwidth. The designed matching circuits are also critical to the system. Without them, the results show that the antennas would not radiate at all. Now, the proposed structure results a good cellular performance, and fulfills the design targets. Generally, the performance of the designed antennas corresponds with the project requirements.

The metal cover makes obtaining good and wideband matching difficult. Large conductive plate at close proximity causes the response to be very resonating with rather narrow bands. As all the antennas are integrated to the side frame of the phone, finding good locations for each of them is quite simple, as is determining the optimal dimensions to have the antennas operating at correct frequencies. Finding suitable matching circuits to reach sufficient performance is the challenge. Based on the results it is understandable why many of the previous studies have slots in the back cover.

Even though this structure performs well, it can be developed further. Antennas for metal-covered handsets are still pretty little researched, especially if compared to mobile antennas in general. Metal-covered phones will become more popular in the future, simply due to their nice outlooks and robustness. Before that can happen, antennas and other communications subsystems must be able to operate according to the requirements by the network standards.

To conclude, the presented antennas operate on all the desired frequencies (LTE, GPS, and Wi-Fi) with good efficiencies, and the cellular antennas are fully MIMO capable. The design objectives are fulfilled, even though the environment is rather challenging. As a final remark, the metal cover clearly affects the antenna performance by lowering efficiency, and making wideband matching quite difficult. Nonetheless, having a mobile phone with full metal housing does not seem impossible based on these results.


\clearpage