%\section{Yhteenveto}
\section{Summary and conclusions} 
\label{sec:conclusions}

\begin{comment}
\begin{itemize}
\item[--]yhteenveto mitä tehty
\item[--]päätulokset
\item[--]johtopäätökset
\item[--]oma arvio?
\end{itemize}
\end{comment}

In this thesis, antennas for metal-covered handsets are studied. The work consists of two parts: literature review on previously studied antennas and a design project. The main objectives of this work are to design antennas for mobile terminal with unbroken metallic back cover, and to understand the challenges in the design process caused by the cover.

As the wireless networks have developed, and the volume of data traffic has simultaneously increased, more and more is required from the mobile terminal. Desires for better robustness and aesthetics by using metal covers significantly increase the complexity of the system in an antenna design point of view. The network requires antennas to communicate efficiently on a wide frequency band, but the metal structures disturb this. Traditional mobile phone antennas, e.g.\ PIFAs, placed inside the phone are highly affected by the surrounding conductive materials.

Majority of the previous studies on this topic have only a metallic side frame, and in the few studies that have also back cover, slots have been cut into it. Antenna structures proposed in previous studied have had decent performance, but the mechanics of them are rather complex. This thesis differs from those studies as the back cover of the handset is a single continuous metal plate. Also, the designed antennas have quite simple structure, and are integrable to the metal frame on the sides of the device.

Considering the challenging environment due to the metal cover, the defined performance targets are quite strict and uneasy to achieve. The two cellular antennas are supposed to operate at $704-960\,\mega\hertz$ and $1.71-2.69\,\giga\hertz$, have at least 30\,\% efficiency, and be fully MIMO capable. Additionally, the two GPS/Wi-Fi antennas should have at least $40\,\%$ efficiency. As a remark, only a minority of previous studies support the $700-800\,\mega\hertz$ cellular band, or MIMO.




\clearpage