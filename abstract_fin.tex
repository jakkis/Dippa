%Tiivistelmässä on lyhyt selvitys (noin 100 sanaa)
%kirjoituksen tärkeimmästä sisällöstä: mitä ja miten on tutkittu,
%sekä mitä tuloksia on saatu. 


Tässä diplomityössä tutkitaan matkapuhelimen yhtenäisen metallikuoren vaikutuksia antennien suorituskykyyn. Tämän lisäksi työssä suunnitellaan LTE-taajuuksilla toimiva MIMO-antenni, sekä GPS- ja Wi-Fi-antennit.

Tutkimus suoritetaan sähkömagneettisilla simulaatioilla. Suunniteltaville antenneille on ennalta määrtätyt suorituskykvaatimukset, jotka vastaavat nykypäivän standardeja. Lisäksi simulaatioissa käytettävä puhelinmalli on ennalta määrätty, sekä paljon yksityiskohtaisempi kuin aiemmissa tutkimuksissa käytetyt mallit. Matkapuhelinantennien tulee toimia taajuuskaistoilla $704-960\,\mega\hertz$ ja $1.71-2.69\,\giga\hertz$ vähintään 30\,\% hyötysuhteella.

Jokaiselle antennille suunnitellaan myös sovituspiiri. Suurin yksittäinen metallikuoresta aiheutuva haaste on riittävän hyvän, laajakaistaisen sovitustason saavuttaminen. Vaikeasta ympäristöstä huolimatta suunnitellut antennit täyttävät niille asetetut vaatimukset.