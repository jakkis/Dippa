%Tiivistelmässä on lyhyt selvitys (noin 100 sanaa)
%kirjoituksen tärkeimmästä sisällöstä: mitä ja miten on tutkittu,
%sekä mitä tuloksia on saatu. 

Tässä diplomityössä tutkitaan matkapuhelimen yhtenäisen metallikuoren vaikutuksia antennien suorituskykyyn. Tämän lisäksi työssä suunnitellaan LTE-taajuuksilla toimiva MIMO-antenni sekä GPS- ja Wi-Fi-antennit. Matkapuhelinantennien tulee toimia taajuuskaistoilla $704-960\,\mega\hertz$ ja $1.71-2.69\,\giga\hertz$ vähintään 30\,\% hyötysuhteella. Vastaavasti muiden antennien taajuuskaistat ovat 1.575\,GHz, 2.4\,GHz ja 5\,GHz tavoitehyötysuhteen ollessa 40\,\%. \\

Tutkimus suoritetaan sähkömagneettisilla simulaatioilla. Työssä käytettävä puhelinmalli on paljon realistisempi kuin aiemmissa tutkimuksissa käytetyt mallit. Simuloinnit keskittyvät antennirakenteisiin, sekä niiden sijainteihin, kokoihin ja syöttöratkaisuihin. Lisäksi jokaiselle antennille suunnitellaan sovituspiiri. Metallikuoren vaikutusta tutkitaan useilla eri testiskenaarioilla, joista jokainen keskittyy yhteen parametriin kuten antennin mittoihin, sijaintiin tai syöttöön. \\

Työ aloitetaan yksinkertaisella puhelinmallilla, ja näiden testien perusteella valitaan puhelinantennirakenne. Lopulliset testit ja antennien optimointi tehdään tarkemmalla mallilla. Esitetty rakenne koostuu kolmesta lähekkäin olevasta ja voimakkaasti kytkevästä antennielementistä. Puhelinantennit sijaitsevat laitteen eri päissä ja ovat osa puhelimen sivuissa olevaa metallirengasta. GPS- sekä Wi-Fi-antennit ovat niin ikään osa metallirengasta, ja sijaitsevat päätyjen väliin jäävällä alueella. \\

Suurin yksittäinen me\-tal\-li\-kuo\-res\-ta aiheutuva haaste on riittävän hyvän ja laajakaistaisen sovitustason saavuttaminen. Vaikeasta ympäristöstä huolimatta suunnitellut antennit täyttävät niille asetetut suorituskykyvaatimukset. Tulosten perusteella on mahdollista suunnitella hyvin toimivat antennit metallikuoriseen puhelimeen.
