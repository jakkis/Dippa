%\section{Tulokset}
\section{Analysis}
\label{sec:analysis}


\subsection{Fulfilment of objectives}
\label{sec:fulfilment}
In this thesis, it was defined to design two cellular antennas to operate on both $704-960\,\mega\hertz$ and $1.71-2.69\,\giga\hertz$ frequency ranges. Additionally, two antennas to support GPS and Wi-Fi connections should be designed. Moreover, each antenna should be matched to $-5\,\db$, and have a certain efficiency. The minimum efficiencies were $30\,\%$ for cellular and $40\,\%$ for other antennas. Also, the isolation between the two cellular antennas should be at least $-15\,\db$.

The final structure indeed has all four required antennas, which are all performing quite well. Obtaining the desired matching level was the most problematic part, and as was seen in the results, the target mostly is not reached. The best matching levels were achieved in the $5\,\giga\hertz$ Wi-Fi band. However, as it was explained earlier, worse matching is accepted if efficiency target is reached. As the efficiency results showed, those targets are reached by cellular antennas, and nearly reached by other antennas. The efficiencies were calculated with (\ref{eq:eff_aprx}), and thus it must be remembered that the results are only approximations. Them can be anyway considered quite accurate as the only lossy part of the simulation model was the plastic rim.

Generally, the design objectives are fulfilled very well, and the antennas are performing as desired. Of course, performance can always be improved, which is discussed further in \ref{sec:improvements}.

\subsection{MIMO capability}
\label{sec:mimo_cap}

\subsection{General discussion}
\label{sec:general_discussion}



\subsection{Possible improvements \& future work}
\label{sec:improvements}



\clearpage